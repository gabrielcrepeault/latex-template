%% Aide-mémoire
\documentclass[10pt, french]{article}
%% -----------------------------
%% Préambule
%% -----------------------------
\usepackage{xcolor}
\usepackage{bbding}
\usepackage{pifont}
% Je met en commentaire, sinon beaucoup trop long à compile
\usepackage{tikz}
%\usepackage{pgfplots}
%\usepackage{array}
\definecolor{LongColor}{HTML}{E61D3B}
\definecolor{ShortColor}{HTML}{2C97D8}
\def\cours{Gestion du risque financier II}
\def\sigle{ACT-2011}
\def\session{Hiver 2018}
\def\auteur{Nicholas Langevin}
\def\BackgroundColor{white}
\def\SectionColor{teal!90!black}
\def\SubSectionColor{teal!45!black}
\input{preamble/cheatsht-preamble.tex}
\title{GRF-II \\ Document d'étude}
\author{Nicholas Langevin}
%% -----------------------------
%% Début du document
%% -----------------------------
\begin{document}

\input{preamble/page-couverture}
\newpage

\small
\begin{multicols*}{3} % Nombre de colonnes (peut être changé plus tard.)

\section{Introduction aux produits dérivés}
\paragraph{Produits dérivés} Contrat entre 2 parties qui fixe les flux financiers futurs fondé sur ceux de \underline{l'actif sous-jacent} $S$.

\subsection*{Étapes d'une transaction}
\begin{enumerate}
\item l'acheteur et le vendeur se trouve (sur un marché quelquonque)
\item on définit les obligations de chaques parties (\textit{i.e. actif à livrer, date d'échéance, prix, etc.}. \textbf{Note : il y a souvent un intermédiaire (\textit{clearing house}) qui intervient.}
\item La transaction a lieu et les obligations sont remplies par chaque parties
\item Les registres de propriétés sont mis à jour.
\end{enumerate}

\paragraph{Transaction gré-à-gré} transaction sans intermédiaire ou à l'extérieur de la bourse. Plusieurs raisons peuvent justifier ce type de transaction  :
\begin{itemize}
\item Ce sont souvent de grosses transaction. On peut donc économiser sur les frais de transaction.
\item On peut combiner (sur une même transaction) plusieurs micro-transaction et plusieurs types d'actifs.
\end{itemize}

\paragraph{Valeur notionelle} \hl{définition exacte à valider}

\paragraph{Origine des marchés de produits dérivés} Après 1971, le président Nixon a vouli défaire le standard de l'or (qui a causé de l'hyperinflation dans plusieurs pays) pour plutôt laisser le libre-marché fixer la valeur des devise de chaque pays.

\paragraph{Rôle des marchés financiers} Partage du risque et diversification des risques.

\paragraph{Utilité des produits dérivés}
\begin{itemize}
\item Gestion des risques
\item Spéculation
\item Réduction des frais de transaction
\item Arbitrage réglementaire
\end{itemize}

\paragraph{Bid-Ask Spread} Correspond à la marge que le teneur de marché (\textit{market maker}) conserve. En l'absence d'arbitrage, on aura $Ask - Bid > 0$
\begin{description}
\item[Ask] prix le plus haut que quelqu'un est prêt à payer pour le sous-jacent
\item[Bid] prix le plus bas que quelqu'un est prêt à payer pour le sous-jacent
\end{description}

\subsection*{Terminologie}
\begin{description}
\item[market order] ordre au marché : on achète et vend selon les prix Bid Ask actuels.
\item[limit order] Ordre limite : on achète le sous-jacent si $Ask < k$ ou on vend le sous-jacent si $Bid > k$.
\item[Stop Loss] ordre de vente stop : on veut limiter sa perte si un sous-jacent perd énormément de valeur. Donc, on va vendre le sous-jacent si $Bid \leq k$.
\item[Long] On se considère en position longue sur le sous-jacent si notre stratégie nous permet de bénéficier d'une \green{hausse} du sous-jacent.
\item[Short] On se considère en position longue sur le sous-jacent si notre stratégie nous permet de bénéficier d'une \red{baisse} du sous-jacent.
\end{description}

\paragraph{Type de risques}
\begin{description}
\item[Risque de défaut] \hl{à préciser}
\item[Risque de rareté] \hl{à préciser}
\end{description}

\section{Introduction aux Forwards et aux options}
Pour chaque stratégie qu'on voit dans le cours, on peut calculer
\begin{description}
\item[Premium] Il s'agit des cashflow à $t=0$ (si positif, il s'agît d'un coût ; si négatif, il s'agît d'une \textit{compensation}).
\item[Payoff] Valeur à l'échéance $t = T$, i.e. les Cash-flow au temps $t = T$.
\item[Profit] $= Payoff - \text{AV}(Premium)\footnote{AV veut dire \textit{accumulated value}.}$
\end{description}

\begin{description}
\item[$r_f$] taux sans risque. Parfois exprimé comme une force d'intérêt $r$ continue.
\item[$S$] Sous-jacent (peut être une action, une devise, ...)
\item[$S_0$] valeur actuelle du sous-jacent $S$.
\item[$S_T$] valeur du sous-jacent $S$ au temps $t = T$.
\item[$F_{0,T}$] Prix \textit{forward} du sous-jacent au temps $T$, qu'on définit comme
\[F_{0,T} = S_0 (1 + r_f)^{T}\]
\item[$F_{0,T}^{P}$] Prix d'un forward prépayé, i.e. on débourse $F_{0,T}^{P}$ à $t=0$ et on reçoit le sous-jacent à $t  = T$, alors
\[F_{0,T}^{P} = F_{0,T} (1 + r_f)^{T}\]
illustration graphique : 
\input{tikz/forward}
\end{description}
\paragraph{Achat ferme et emprunt} On utilise parfois la lettre $S$ pour désigner dans stratégie l'action de faire un achat ferme (i.e. acheter et se faire livrer le sous-jacent à $t=0$) et $B$ pour désigner un dépôt/emprunt (qu'on exprime comme une obligation zéro-coupon).


\subsection*{$Call(K,T)$}
Contrat qui \textit{permet} au détenteur de se procurer $S$ au prix $K$ à l'échéance $T$. \textbf{position longue dans le sous-jacent}
\[Premium = C(K,T) \]
\[Payoff =
\begin{cases}
0				& , S_T \leq K \\
S_T - K		& , S_T > K \\
\end{cases}
\]
\input{tikz/call}


\subsection*{$Put(K,T)$}
Contrat qui \textit{permet} au détenteur de vendre $S$ au prix $K$ à l'échéance $T$. \textbf{position courte dans le sous-jacent}
\[Premium = P(K,T)\]
\[Payoff = 
\begin{cases}
K - S_T			& , S_T \leq K \\
0					& , S_T > K \\
\end{cases}
\]
\input{tikz/put}

\subsection*{Forward synthétique}
On peut créer un Forward synthétique 2 de façon (en combinant d'autres transactions) : 
\[Forward = Stock - Bond \]
\[Forward = Call(K,T) - Put(K,T) \]



\section{Stratégie de couverture}
\subsection*{$Floor$}
On achète $S$ en se protégant contre une baisse trop importante du sous-jacent (\textbf{position longue})
\[Floor = Stock + Put(K,T)\]
\[Premium = S_0 + P(K,T) > 0\]
\[Payoff = 
\begin{cases}
K					& , S_T \leq K \\
S_T				& , S_T > K \\
\end{cases}
\]
\input{tikz/floor}

\subsection*{Cap}
On vend à découvert $S$ en se protégant contre une hausse trop importante du sous-jacent (car il faudra éventuellement le racheter!). \textbf{Position courte}.
\[Cap = Call(K,T) - Stock\]
\[Premium = C(K,T) - S_0 < 0\]
\[Payoff
\begin{cases}
- S_T 			& , S_T \leq K \\
- K				& , S_T > K \\
\end{cases}
\]
\input{tikz/cap}

\subsection*{Bull Spread}
Combinaison de 2 Call (ou 2 Put) pour spéculer sur un marché haussier. Avec $K_1 < K_2$, on a
\paragraph{Avec option d'achat}
\[Bull (Call) = Call(K_1, T) - Call(K_1, T) \]

\[Premium = C(K_1, T) + Call(K_2, T) > 0 \]
\[Payoff = 
\begin{cases}
0					& , S_T \leq K_1 \\
S_T - K_1		& , k_1 < S_T \leq K_2 \\
K_2 -K_1		& , S_T > K_2 \\
\end{cases}
\]
\paragraph{Avec option de vente}
\[Bull(Put) = Put(K_1, T) - Put(K_2,T) \]
\[Premium = P(K_1,T) - P(K_2, T) < 0\]
\[Payoff = 
\begin{cases}
K_1 - K_2 			& , S_T \leq K_1 \\
K_2 - S_T				& , K_1 < S_T \leq K_2 \\
0							& , S_T > K_2 \\
\end{cases}
\]

\input{tikz/bull_spread}


\subsection*{Bear Spread}
Combinaison de 2 Call ou 2 Put pour spéculer sur un marché baissier.
\paragraph{Avec option d'achat}
\begin{align*}
Bear(Call) & =  - Bull(Call) \\
& = Call(K_2, T) - Call(K_1, T) \\
Premium		& = 	 C(K_2,T) - C(K_1,T) < 0 \\
Profit	& = 
\begin{cases}
0					& , S_T \leq K_1 \\
K_1 - S_T 	& , K_1 < S_T \leq K_2 \\
-(K_2 - K_1)& , S_T > K_2 \\
\end{cases}
\end{align*}

\paragraph{Avec option de vente}
\begin{align*}
Bear(Put) 				& =  - Bull(Put) \\
								& = Put(K_2, T) - Put(K_1, T) \\
Premium					& = 	 P(K_2,T) - P(K_1,T) > 0 \\
Profit						& = 
\begin{cases}
K_2 - K_1					& , S_T \leq K_1 \\
K_2 - S_T 				& , K_1 < S_T \leq K_2 \\
0								& , S_T > K_2 \\
\end{cases}
\end{align*}
\input{tikz/bear_spread}

\subsection*{Ratio Spread}
Cette stratégie est une combinaison un peu sur mesure (on ne peut pas nécessairement dire si elle est longue ou courte). On achète $n$ options d'achat à un prix d'exercice $K_1$ et on en vend $m$ à un prix d'exercice $K_2$.\footnote{On peut faire cette stratégie avec des options de vente aussi.}
\begin{align*}
Ratio Spread 		& 	= n Call(K_1, T) - m Call(K_2, T) \\
Premium				& = n C(K_1, T) - m C(K_2,T) \\
Payoff					& = ...
\end{align*}

\subsection*{Box Spread}
Cette stratégie réplique l'achat d'une obligation zéro-coupon, en impliquant 2 option d'achat et 2 options de vente.
\begin{align*}
Box Spread			& = Bull(Call) + Bear(Put) \\
& = Call(K_1, T) - Call(K_2, T)  \\
& + Put(K_2, T) - Put(K_1, T) \\
Premium				& = C(K_1, T) - C(K_2, T) \\
							&  + P(K_2, T) - P(K_1,T) >  0 \\
Payoff					& = K_2 - K_1  \ , \forall S_T \\
\end{align*}
\input{tikz/box_spread}


\subsection*{Collar}
La prime initiale du Collar peut être soit positive ou négative (dépendant du strike price).
\begin{align*}
Collar				& = Put(K_1, T) - Call(K_2, T) \\
Premium			& = P(K_1, T) - C(K_2,T) \\
Payoff				& = 
\begin{cases}
K_1 - S_T			& , S_T \leq K_1 \\
0						& , K_1 < S_T \leq K_2 \\
K_2 - S_T			& , S_T > K_2 \\
\end{cases}
\end{align*}
\input{tikz/collar}


\subsection*{Stock Covered by Collar}
\begin{itemize}
\item On effectue la même stratégie qu'un Collar, en ayant initialement le sous-jacent $S$. \textbf{Position longue dans le sous-jacent}.
\item Cette stratégie reproduit les flux monétaires d'un Bull Spread, alors
\end{itemize}
\begin{align*}
BullSpread				& = Collar + Stock \\
								& = Put(K_1, T) - Call(K_2, T) + Stock \\
Premium					& = P(K_1, T) - C(K_2,T) + S_0 > 0  \\
Payoff						& = 
\begin{cases}
K_1 					& , S_T \leq K_1 \\
S_T					& , K_1 < S_T \leq K_2 \\
K_2 					& , S_T > K_2 \\
\end{cases}
\end{align*}
\input{tikz/stock_covered_collar}


\subsection*{Straddle}
Stratégie pour spéculer sur la volatilité du sous-jacent $S$ autour du point $K$.
\begin{align*}
Straddle			& = Put(K,T) + Call(K,T) \\
Premium			& = P(K,T) + C(K,T) > 0 \\
Payoff				& =
\begin{cases}
K - S_T				& , S_T \leq K \\
S_T - K				& , S_T > K \\
\end{cases}
\end{align*}
\input{tikz/straddle}

\subsection*{Strangle}
Même genre de stratégie que le strangle, on spécule sur la volatilité du sous-jacent à l'extérieur de l'intervalle $[K_1, K_2]$ : 
\begin{align*}
Strangle			& = Put(K_1, T) + Call(K_2, T) \\
Premium			& = P(K_1, T) + C(K_2, T) > 0 \\
Payoff				& = 
\begin{cases}
K_1 - S_T			& , S_T \leq K_1 \\
0						& , K_1 < S_T \leq K_2 \\
S_T - K_2 		& , S_T > K_2 \\
\end{cases}
\end{align*}
\input{tikz/strangle}


\subsection*{Butterfly Spread (BFS)}
On combine un Straddle($K_2$) et un Strangle($K_1, K_3$) pour spéculer sur la non-volatilité du sous-jacent autour de $K_2$, mais en limitant nos pertes à $K_1 - K_2$ : 
\begin{align*}
Butterfly  & = Strangle - Straddle(K_2) \\
& = Put(K_1, T) - Put(K_2, T) \\
& - Call(K_2, T) + Call(K_3,T) \\
Premium	& = P(K_1, T) - P(K_2, T) \\
& - C(K_2, T) + C(K_3, T) < 0\\
Payoff		& = 
\begin{cases}
K_1 - K_2			& , S_T \leq K_1 \\
S_T - K_2			& , K_1 < S_T \leq K_2 \\
K_2 - S_T			& , K_2 < S_T \leq K_3 \\
K_2 - K_3			& , S_T > K_3 \\
\end{cases}
\end{align*}
\paragraph{Note} De façon générale (plusieurs combinaisons sont possibles), on a 
\[BFS = Bull(K_1, K_2) + Bear(K_2, K_3) \]
\input{tikz/butterfly_spread}

\paragraph{Asymetric Butterfly Spread} 
\begin{itemize}
\item Comme le Ratio Spread, il est possible de faire une stratégie sur mesure en achetant $n$ Bull Spread et en achetant $m$ Bear Spread en respectant les 3 prix d'exercices $K_1 < K_2 < K3$.
\item Si on désire avoir un BFS qui a un profit nul pour $S_T < K_1$ et $S_T > K_3$, alors on trouve $n$ et $m$ tel que
\begin{align*}
\frac{n}{m} = \frac{K_3 - K_2}{K_2 - K_1}
\end{align*}
\end{itemize}

% --- Chapitre 5 ---
\setcounter{section}{4}
\section{Forwards et Futures}






% --- Chapitre 9 ---
\setcounter{section}{8}
\section{Put-Call Parity}
définition de base : 
\[C(K,T) - P(K,T) = F_{0,T} - K(1+r_f)^{T}\]
dans le cas où l'action verse des dividendes : 
\begin{align*}
C(K,T) - P(K,T) 		& = S_0 - \text{PV}(Div) - K(1 + r_f)^{T} \\
& = S_0 e^{-\delta T} - K e^{-rT}
\end{align*}
Dans le cas où le sous-jacent en question est une devise étrangère (DÉ) qu'on achète avec notre devise locale (DD) : 






\end{multicols*}

%% -----------------------------
%% Fin du document
%% -----------------------------
\end{document}