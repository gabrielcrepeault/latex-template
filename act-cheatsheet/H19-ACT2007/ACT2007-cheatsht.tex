\documentclass[10pt, french]{article}

% Fichier préambule contenant la configuration pour la feuille aide-mémoire
\input{preamble/cheatsht-preamble.tex}

\setlength{\abovedisplayskip}{-15pt}
\setlength{\belowdisplayskip}{0pt}
\setlength{\abovedisplayshortskip}{0pt}
\setlength{\belowdisplayshortskip}{0pt}


\begin{document}
% j'enlève le footnotesize temporairement, sinon je ne vois rien! GCC
% \footnotesize % Écrire petit (peut être modifié)
\begin{multicols*}{3} % Permet d'écrire dans plusieurs colonnes l'entièreté du document
% -------------------------------------------------------------
\section{Rappel de Math-vie 1}
\hl{Mettre formule des assurances et rentes importantes}
\subsection*{Modèles de survie}
\begin{description}
\item[$T_x$] durée de vie de $(x)$
\end{description}



\subsection*{Contrat d'assurance}

\subsection*{Contrat de rente}
\[\ax**{x} = \frac{1 - \Ax{x:\angln}}{d} \leftrightarrow \Ax{\endowxn} =1 - d \ax{x}\]




\subsection*{Principe d'équivalence}


\section{Calcul de réserve}
\paragraph{Perte prospective}
\begin{align*}
\actsymb[t]{L}{} & = \{ \actsymb[t]{L}{} | T_x > t \} \\
& = VP_{@t}(\text{Prest.}) - VP_{@t}(\text{Primes}) \\
& = Z - Y
\end{align*}

\paragraph{Réserve au temps $t$}
\begin{align*}
\actsymb[t]{V}{} = \esp{\actsymb[t]{L}{}}
\end{align*}











% -------------------------------------------------------------
% Fin de la feuille aide-mémoire
\end{multicols*}
\end{document}
