\documentclass[10pt, french]{article}

% Fichier préambule contenant la configuration pour la feuille aide-mémoire
\input{preamble/cheatsht-preamble.tex}

\setlength{\abovedisplayskip}{-15pt}
\setlength{\belowdisplayskip}{0pt}
\setlength{\abovedisplayshortskip}{0pt}
\setlength{\belowdisplayshortskip}{0pt}


\begin{document}
% j'enlève le footnotesize temporairement, sinon je ne vois rien! GCC
% \footnotesize % Écrire petit (peut être modifié)
\begin{multicols*}{3} % Permet d'écrire dans plusieurs colonnes l'entièreté du document
% -------------------------------------------------------------
\section{Rappel de Math-vie 1}
\subsection*{Modèles de survie}
\begin{itemize}
\item $T_x$ : durée de vie de $(x)$
\item $K_x =  \lfloor T_x \rfloor$
\item $\px[t]{x}[] = \prob{T_x > t} = e^{- \int_{0}^{t} \mu_{x+s} ds}$
\item $\qx[t]{x}[] = 1 - \px[t]{x}[] = \prob{T_x \leq t}$
\item $\px[t + u]{x}[] = \px[t]{x}[] \cdot \px[u]{x+t}[]$
\item $\qx[t|u]{x}[] = \px[t]{x}[] \cdot \qx[u]{x+t}[]$
\end{itemize}

\subsection*{Contrat d'assurance}
\paragraph{Assurance entière}
\[\Ax{x} = \sum_{k=0}^{\infty} b_k v^{k+1} \qx[k|]{x}[]\]
\paragraph{Assurance dotation pure (\emph{pure endowment)}}
\[\Ax{\endowxn} = \Ax{\termxn} + \Ax{\pureendowxn}\]
où $\Ax{\pureendowxn} = \Ex[n]{x} = v^n \px[n]{x}[]$

\paragraph{Assurance temporaire $n$ année}
\[\Ax{\termxn} = \Ax{x} - \Ax[n|]{x} \]
où $\Ax[n|]{x}  = \Ex[n]{x} \Ax{x+n}$ (i.e. une assurance différée)

\paragraph{Assurance payable $m$ fois l'an}
\[\Ax{x}[(m)] = \sum_{k=0}^{\infty} v^{\frac{(k+1)}{m}} \qx[\frac{k}{m} | \frac{1}{m}]{x}[] \]


\subsection*{Contrat de rente}
\paragraph{Rente entière}
\[\ax**{x} = \sum_{k=0}^{\infty} v^{k} \px[k]{x}[]  \]
\[\ax**{x} = \frac{1 - \Ax{x:\angln}}{d} \leftrightarrow \Ax{\endowxn} =1 - d \ax{x}\]




\subsection*{Principe d'équivalence}
$\pi$, lorsque calculée sous le principe d'équivalence, est la solution de
\[\esp{Z} = \esp{Y}\]
où $Z$ est la valeur présente des prestations futures et $Y$ la valeur présente des primes futures à recevoir.

\section{Calcul de réserve}
\paragraph{Perte prospective}
\begin{align*}
\actsymb[t]{L}{} & = \{ \actsymb[t]{L}{} | T_x > t \} \\
& = VP_{@t}(\text{Prest.}) - VP_{@t}(\text{Primes}) \\
& = Z - Y
\end{align*}

\paragraph{Réserve au temps $t$}
Selon la méthode prospective,
\[\actsymb[t]{V}{} = \esp{\actsymb[t]{L}{}} = \esp{Z} - \esp{Y}\]
Selon la méthode rétrospective,
\[\actsymb[t]{V}{} = \frac{\text{VPA}_{@t}(\text{$\pi$ reçues avant $h$}) - \text{VPA}_{@t}(\text{Prest. à payer avant $h$})}{g}\]

\paragraph{Relation récursive pour les réserves (discrètes, sans frais)}
\[\actsymb[h+1]{V}{} = \frac{(\actsymb[h]{V}{} + \pi_h)(1+i) - b_{h+1} \qx[]{x+h}[]}{\px[]{x+h}[]}\]








% -------------------------------------------------------------
% Fin de la feuille aide-mémoire
\end{multicols*}
\end{document}
