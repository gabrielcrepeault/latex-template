\documentclass[10pt, french]{article}

%% -----------------------------
%% Préambule
%% -----------------------------
% !TEX encoding = UTF-8 Unicode
% LaTeX Preamble for cheatsheet
% Author : Gabriel Crépeault-Cauchon

% HOW-TO : copy-paste this file in the same directory as your .tex file, and add in your preamble the next command right after you have specified your documentclass :
% \input{preamble-cheatsht.tex}
% ---------------------------------------------
% ---------------------------------------------

% Extra note : this preamble creates document that are meant to be used inside the multicols environment. See the documentation on internet for further information.

%% -----------------------------
%% Encoding packages
%% -----------------------------
\usepackage[utf8]{inputenc}
\usepackage[T1]{fontenc}
\usepackage{babel}
\usepackage{lmodern}

%% -----------------------------
%% Variable definition (change these values)
%% -----------------------------
\def\cours{Analyse probabiliste des risques actuariels}
\def\sigle{ACT-1002}
\def\session{H2017}
\def\auteur{Gabriel Crépeault-Cauchon}
\def\BackgroundColor{white}
\def\SectionColor{teal!50!black}
\def\SubSectionColor{teal!20!black}

%% -----------------------------
%% Margin and layout
%% -----------------------------
% Determine the margin for cheatsheet
\usepackage[landscape, hmargin=1cm, vmargin=1.7cm]{geometry}
\usepackage{multicol}

% Remove automatic indentation after section/subsection title.
\setlength{\parindent}{0cm}

% Save space in cheatsheet by removing space between align environment and normal text.
\usepackage{etoolbox}
\newcommand{\zerodisplayskips}{%
  \setlength{\abovedisplayskip}{0pt}%
  \setlength{\belowdisplayskip}{0pt}%
  \setlength{\abovedisplayshortskip}{0pt}%
  \setlength{\belowdisplayshortskip}{0pt}}
\appto{\normalsize}{\zerodisplayskips}
\appto{\small}{\zerodisplayskips}
\appto{\footnotesize}{\zerodisplayskips}

%% -----------------------------
%% URL and links
%% -----------------------------
\usepackage{hyperref}
\hypersetup{colorlinks = true, urlcolor = gray!70!white, linkcolor = black}

%% -----------------------------
%% Document policy (uncomment only one)
%% -----------------------------
%	\usepackage{concrete}
	\usepackage{mathpazo}
%	\usepackage{frcursive} %% permet d'écrire en lettres attachées
%	\usepackage{aeguill}
%	\usepackage{mathptmx}
%	\usepackage{fourier}

%% -----------------------------
%% Math configuration
%% -----------------------------
\usepackage[fleqn]{amsmath}
\usepackage{amsthm,amssymb,latexsym,amsfonts}
\usepackage{empheq}
\usepackage{numprint}

% Mathematics shortcut
\newcommand{\reels}{\mathbb{R}}
\newcommand{\entiers}{\mathbb{Z}}
\newcommand{\naturels}{\mathbb{N}}
\newcommand{\eval}{\biggr \rvert}
\usepackage{cancel}
\newcommand{\derivee}[1]{\frac{\partial}{\partial #1}}
\newcommand{\prob}[1]{\Pr \left( #1 \right)}
\newcommand{\esp}[1]{\mathrm{E} \left[ #1 \right]} % espérance
\newcommand{\variance}[1]{\mathrm{Var} \left( #1   \right)}
\newcommand{\covar}[1]{\mathrm{Cov} \left( #1   \right)}
\newcommand{\laplace}{\mathcal{L}}

% To indicate equation number on a specific line in align environment
\newcommand\numberthis{\addtocounter{equation}{1}\tag{\theequation}}

% Actuarial notation package
\usepackage{actuarialsymbol}
\usepackage{actuarialangle}

% Matricial anotation for math symbols (\bm{•})
\usepackage{bm}
% matricial notation variable (bold style)
\newcommand{\matr}[1]{\mathbf{#1}}



%% -----------------------------
%% tcolorbox configuration
%% -----------------------------
\usepackage{tcolorbox}
\tcbuselibrary{xparse}
\tcbuselibrary{breakable}

%% Définition boite pour définition
\DeclareTColorBox{definition}{ o }% #1 parameter
{colframe=blue!60!green,colback=blue!5!white, % color of the box
breakable, pad at break*=0mm, % to split the box
title = {#1},
after title = {\large \hfill \faBook}
}

%% -----------------------------
%% Graphics and pictures
%% -----------------------------
\usepackage{graphicx}
\usepackage{pict2e}

%% -----------------------------
%% insert pdf pages into document
%% -----------------------------
\usepackage{pdfpages}

%% -----------------------------
%% Color configuration
%% -----------------------------
\usepackage{color, soulutf8, colortbl}

% usefull shortcut for colored text
\newcommand{\orange}{\textcolor{orange}}
\newcommand{\red}{\textcolor{red}}
\newcommand{\cyan}{\textcolor{cyan}}
\newcommand{\blue}{\textcolor{blue}}
\newcommand{\green}{\textcolor{green}}
\newcommand{\purple}{\textcolor{magenta}}
\newcommand{\yellow}{\textcolor{yellow}}


%% -----------------------------
%% Enumerate environment configuration
%% -----------------------------
% Custum enumerate & itemize Package
\usepackage{enumitem}
% French Setup for itemize function
\frenchbsetup{StandardItemLabels=true}
% Change default label for itemize
\renewcommand{\labelitemi}{\faAngleRight}

%% -----------------------------
%% Tabular column type configuration
%% -----------------------------
\newcolumntype{C}{>{$}c<{$}} % math-mode version of "l" column type
\newcolumntype{L}{>{$}l<{$}} % math-mode version of "l" column type
\newcolumntype{R}{>{$}r<{$}} % math-mode version of "l" column type
\newcolumntype{f}{>{\columncolor{green!20!white}}p{1cm}}
% configuration to force a line break within a single cell
\usepackage{makecell}



%% -----------------------------
%% Fontawesome for special symbols
%% -----------------------------
\usepackage{fontawesome}

%% -----------------------------
%% Section Font customization
%% -----------------------------
\usepackage{sectsty}
\sectionfont{\color{\SectionColor}}
\subsectionfont{\color{\SubSectionColor}}

%% -----------------------------
%% Footer/Header Customization
%% -----------------------------
\usepackage{lastpage}
\usepackage{fancyhdr}
\usepackage{titling}
\pagestyle{fancy}
% Header
\fancyhead{} 	% Reset
\fancyhead[L]{Aide-mémoire pour~ \cours ~(\textbf{\sigle})}
\fancyhead[R]{\auteur}

% Footer
\fancyfoot{}		% Reset
\fancyfoot[R]{\thepage ~de~ \pageref{LastPage}}
\fancyfoot[L]{\href{https://github.com/gabrielcrepeault/latex-template}{\faGithub \ gabrielcrepeault/latex-template}}

% page background color
\pagecolor{\BackgroundColor}






%% END OF PREAMBLE
% ---------------------------------------------
% ---------------------------------------------


% On débute la numérotation des chapitres à 2 pour suivre les notes à Marie-Pier.
\setcounter{section}{1}


%% -----------------------------
%% Début du document
%% -----------------------------
\begin{document}

% \small
\begin{multicols*}{3} % Nombre de colonnes (peut être changé plus tard.)
\section{Régression linéaire simple}
\subsection*{Postulats}
\begin{enumerate}[label=$\mathbf{H}_{\arabic*}$]
\item Linéarité : $\esp{\varepsilon_i} = 0$
\item Homoscédasticité : $\variance{\varepsilon_i}= \sigma^2$
\item Indépendance : $\covar{\varepsilon_i, \varepsilon_j} = 0$
\item Normalité : $\varepsilon_i \sim \mathcal{N} (0, \sigma^2)$
\end{enumerate}
\subsection*{Modèle}
\begin{align*}
\esp{Y_i | x_i} 	& = \beta_0 + \beta_1 x_i \\
\variance{Y_i | x_i}	& = \sigma^2 \\
Y_i | x_i & \overset{\mathbf{H}_4}{\sim} \mathcal{N} (\beta_0 + \beta_1 x_i, \sigma^2) \\
\end{align*}

\subsection*{Estimation des paramètres}
\begin{align*}
\hat{\beta}_0 	& = \bar{Y} - \hat{\beta}_1 \bar{x} \\
\hat{\beta}_1	& = \frac{\sum_{i=1}^{n} x_i Y_i - \bar{Y} \sum_{i=1}^{n} x_i}{\sum_{i=1}^{n} x_i^2 - \bar{x} \sum_{i=1}^{n} x_i} = \frac{\sum_{i=1}^{n}(x_i - \bar{x})Y_i}{S_{XX}}
\end{align*}

\subsection*{Estimation de $\sigma^2$}
\begin{align*}
\hat{\sigma^2} = s^2 = \frac{\sum_{i=1}^{n} \hat{\varepsilon_i}^2}{n-p'} = \frac{\sum_{i=1}^{n} (Y_i - \hat{Y}_i)^2}{n-2}
\end{align*}

\subsection*{Propriété des estimateurs}
\begin{align*}
\esp{\hat{\beta}_1} = \beta_1 \quad , \variance{\hat{\beta}_1} = \frac{\sigma^2}{S_{XX}} \\
\hat{\beta}_1 \overset{H_4}{\sim} \mathcal{N}(\beta_1, \frac{\sigma^2}{S_{XX}}) \\
\esp{\hat{\beta}_0} = \beta_0 \quad , \variance{\hat{\beta}_0} = \sigma^2 \left( \frac{1}{n} + \frac{\bar{x}^2}{S_{XX}} \right) \\
\hat{\beta}_0 \overset{H_4}{\sim} \mathcal{N} \left(\beta_0, \sigma^2 \left( \frac{1}{n}  + \frac{\bar{x}^2}{S_{XX}} \right) \right) \\
\mathrm{Cov}(\hat{\beta}_0, \hat{\beta}_1) = - \frac{\bar{x} \sigma^2}{S_{XX}}
\end{align*}

\subsection*{Tests d'hypothèse sur les paramètres}
$H_0$ : $\hat{\beta} = \theta_0$ , $H_1$ : $\hat{\beta} \neq \theta_0$
\begin{align*}
t_{obs} = \frac{\hat{\beta} - \theta_0}{\sqrt{\hat{Var(\hat{\beta}})}} \sim T_{n-2}
\end{align*}
On Rejette $H_0$ si $t_{obs} > | t_{n-2} (1 - \frac{\alpha}{2})|$

\subsection*{Intervalle de confiance}
\subsubsection*{Pour la droite de régression ($\esp{Y_0|x_0}$)}
Sachant que $\esp{Y_0 | x_0} = \beta_0 + \beta_1 x_0$, on a l'IC suivant
\begin{align*}
\left[ \hat{Y_0} \pm t_{n-2} \left(1 - \frac{\alpha}{2} \right) \sqrt{s^2 \left( \frac{1}{n} + \frac{(x_0 - \bar{x})^2}{S_{XX}} \right)} \right]
\end{align*}

\subsubsection*{Pour la prévision de $Y_0$}
Sachant que $Y_0 = \beta_0 + \beta_1 x_0 + \varepsilon$, on a l'IC suivant
\begin{align*}
\hat{Y_0} \pm t_{n-2} \left(1 - \frac{\alpha}{2} \right) \sqrt{s^2 \left( 1 + \frac{1}{n} + \frac{(x_0 - \bar{x})^2}{S_{XX}} \right)}
\end{align*}

\subsection*{Analyse de la variance (ANOVA)}
\begin{tabular}{|f| *{4}{C|}}
\hline
\rowcolor{green!40!black} Source & \text{dl} & \text{SS} & \text{MS} & F \\\hline
Model & p & \thead{\sum_{i=1}^{n} (\hat{Y}_i - \bar{Y})^2 \\ (SSR)}  & \thead{SSR / dl_1 \\ (MSR)} & \frac{MSR}{MSE} \\\hline
Residual error & n - p' & \thead{\sum_{i=1}^{n} (Y_i - \hat{Y}_i)^2 \\ (SSE)} & \thead{SSE / dl_2 \\(MSE = s^2)} & \cellcolor{gray!30!white} \\\hline
Total & n - 1 & \thead{\sum_{i=1}^{n} (Y_i - \bar{Y})^2 \\ (SST)} & \cellcolor{gray!30!white} & \cellcolor{gray!30!white} \\\hline
\end{tabular}
Où $p$ est le nombre de variables explicatives dans le modèle.

\subsubsection*{Coefficient de détermination}
\begin{align*}
R^2 = \frac{SSR}{SST} = 1 - \frac{SSE}{SST}
\end{align*}
On a aussi la relation suivante avec $F_{obs}$ : 
\begin{align*}
F = \frac{R^2}{1 - R^2} \cdot \frac{n-p'}{p}
\end{align*}

\subsubsection*{Test F de Fisher pour la validité globale de la régression}
On rejette $H_0 : \beta_1 =  \beta_2 = ... =  \beta_p = 0$ si 
\begin{align*}
F_{obs} = \frac{MSR}{MSE} \geq F_{n, n-p'}(1 - \alpha)
\end{align*}
où $p$ est le nombre de variables explicatives dans le modèle (régression linéaire simple, $p=1$ et $p' = p+1$).



\subsection*{Distribution d'un résidu $\varepsilon$}
\begin{align*}
\esp{\hat{\varepsilon}_i} = 0 \quad , Var \left( \hat{\varepsilon}_i \right) = \sigma^2 (1 - h_{ii})
\end{align*}
où $h_{ii} = \frac{1}{n} + \frac{(\bar{x} - x_i)^2}{S_{XX}}$.

On peut aussi prouver que
\begin{align*}
\mathrm{Cov}(\hat{\varepsilon}_i, \hat{\varepsilon}_j) = - \sigma^2 \left( \frac{1}{n} + \frac{(x_i - \bar{x})(x_j - \bar{x})}{S_{XX}} \right)
\end{align*}

\subsection*{Vérification des postulats}
Les résidus studentisés sont définis par
\begin{align*}
r_i = \frac{\hat{\varepsilon}_i}{\sqrt{s^2(1 - h_{ii})}}
\end{align*}
\subsubsection*{Linéarité}
\begin{itemize}
\item graphique $Y_i | x_i$
\item graphique $\hat{\varepsilon}_i | \hat{Y}_i$
\item graphique $\hat{\varepsilon}_i | x_i$
\end{itemize}
Les deux derniers graphique doivent être centrés à 0 et d'allure aléatoire.

\subsubsection*{Homoscédasticité}
\begin{itemize}
\item Graphique $r_i | \hat{Y}_i$ : la dispersion des résidus doit être constante, pas de forme d'entonnoir ou de résisus absolus supérieurs à 3.
\end{itemize}

\subsubsection*{Indépendance}
\begin{itemize}
\item Graphique $r_i|i$ : si il y a un \textit{pattern}, présence d'auto-corrélation (le postulat $H_3$ n'est donc pas respecté).
\end{itemize}

\subsubsection*{Normalité}
\begin{itemize}
\item Histogramme des $r_i$
\item Q-Q Plot Normal : les résidus du modèle doivent suivre la droite des quantiles normaux théoriques.
\end{itemize}




\section{Régression linéaire multiple}
% \mathrm{•} utilise la notation matricielle
% \bm{•} est utilisé pour la notation matricielle, mais lorsqu'il y a des symboles mathématiques.
\subsection*{Le modèle et ses propriétés}
\begin{align*}
\matr{Y}_{n \times 1} & = \matr{X}_{n \times p'} \bm{\beta}_{p' \times 1} + \bm{\varepsilon}_{n \times 1} \\
\esp{\matr{Y}}	& = \matr{X} \bm{\beta} \quad , \variance{\matr{Y}} = \sigma^2 \matr{I}_{n \times n} \\
Y & \overset{H_4}{\sim} \mathcal{N}_n(\matr{X} \bm{\beta}, \sigma^2 \matr{I}_{n \times n}) \\
\end{align*}

\subsection*{Paramètres du modèle}
\subsubsection*{Estimation et propriétés des paramètres}
\begin{align*}
\hat{\bm{\beta}} & = \matr{(X^\top X)^{-1} X^\top Y} \\
\esp{\hat{\bm{\beta}}}	& = \bm{\beta} \quad , Var(\hat{\bm{\beta}}) = \sigma^2 (\matr{X^\top X)^{-1}} \\
\hat{\bm{\beta}} & \overset{H_4}{\sim} \mathcal{N}_p(\bm{\beta}, \sigma^2 (\matr{X^\top X)^{-1}})
\end{align*}

\subsubsection*{Intervalle de confiance sur les paramètres}
\begin{align*}
&var[\beta_j] = \sigma^2 v_{jj} \\
&\beta_j \in \left[ \hat{\beta_j} \pm t_{n-p'} \left(1- \frac{\alpha}{2} \right) \sqrt{s^2 v_{jj}} \right]
\end{align*}
où $v_{jj}$ est l'élément $(j,j)$ de la matrice $\matr{(X^\top X)^{-1}}$.

\subsubsection*{Estimation de $\sigma^2$}
\begin{align*}
\hat{\sigma}^2 = s^2 = \frac{\hat{\bm{\varepsilon}}^\top \hat{\bm{\varepsilon}}}{n-p'} \\
\end{align*}
Il peut être démontré que cette estimateur est sans biais \\
et indépendant de $\bm{\hat{\beta}}$

\subsubsection*{Test d'hypothèse sur un paramètre du modèle}
On rejète $H_0 : \beta_j = 0$ si
\begin{align*}
|t_{obs, j}| = \frac{\beta_j}{\sqrt{s^2 v_{jj}} } > t_{n-p'}\left(1 - \frac{\alpha}{2} \right) \\
\end{align*}

\subsection*{Propriétés de la droite de régression}
\begin{align*}
	\hat{\matr{Y}}	& = \matr{X} \bm{\hat{\beta}}  &&\hat{\varepsilon} = \matr{Y} - \hat{\matr{Y}}\\
		& = \matr{X(X^\top X)^{-1} X^\top Y} && \:\:= (\matr{I}_n - \matr{H})\matr{Y}	\\
		& = \matr{H Y} 
\end{align*}	
où $\matr{H = X(X^\top X)^{-1}X^\top}$ est la \textit{hat matrix}.

On a aussi que
\begin{align*}
\esp{\hat{\matr{Y}}} & = \matr{X} \bm{\beta} \quad , \variance{\hat{\matr{Y}}} = \sigma^2 \matr{H} \\
\hat{\matr{Y}} & \overset{H_4}{\sim} N_n(\matr{X} \bm{\beta} , \sigma^2 \matr{H}) \\
\end{align*}

Pour les résidus de la droite de régression, on a
\begin{align*}
\esp{\hat{\bm{\varepsilon}}}  \overset{H_1}{=} 0 \quad , \variance{\hat{\bm{\varepsilon}}} = \sigma^2(\matr{I}_{n \times n} - \matr{H}) \\
\hat{\bm{\varepsilon}} \overset{H_4}{\sim} \mathcal{N}_n(0, \sigma^2 (\matr{I}_{n \times n} - \matr{H})) \\
\end{align*}

\subsection*{Matrice de projection}
Les matrices $\matr{H}$ et $\matr{I_n - H}$ peuvent être vues commes des matrices de projection. Ces deux opérateurs possèdent plusieurs propriétés:
\begin{enumerate}
	\item $\matr{H}^\top = \matr{H}\:$(symétrie)
	\item $\matr{H}\matr{H} = \matr{H}\:$(idempotence)
	\item $\matr{HX} = \matr{X}$
	\item $(\matr{I}_n - \matr{H}) = (\matr{I}_n - \matr{H})^\top \:$(symétrie)
	\item $(\matr{I}_n - \matr{H})(\matr{I}_n - \matr{H}) = (\matr{I}_n - \matr{H})$
	\item $(\matr{I}_n - \matr{H})\matr{X} = 0 $
	\item $(\matr{I}_n - \matr{H})\matr{H} = 0 $
\end{enumerate}

\subsection*{Intervalle de confiance pour la prévision}
\subsubsection*{Théorème de Gauss-Markov}
Selon les postulats $H_1$ à $H_4$, l'estimateur
\begin{align*}
\mathrm{a^\top} \hat{\bm{\beta}} = \matr{a^\top (X^\top X)^{-1} X^\top Y}
\end{align*}
est le meilleur estimateur pour $\matr{a^\top} \bm{\beta}$

(BLUE : \textit{Best linear unbiaised estimator}).

\subsubsection*{I.C. pour la prévision de la valeur moyenne $\esp{\matr{Y} | \matr{X^*}}$}
\begin{align*}
\left[ \matr{{X^*}^\top} \hat{\bm{\beta}} \pm t_{n-p'}\left(1 - \frac{\alpha}{2} \right) \sqrt{s^2 \matr{{X^*}^\top} (\matr{X}^\top \matr{X})^{-1}\matr{X^*}} \right]
\end{align*}

\subsubsection*{I.C. pour la valeur prédite $\hat{\matr{Y}}|\matr{X^*}$}
\begin{align*}
\left[ \matr{{X^*}^\top} \hat{\bm{\beta}} \pm t_{n-p'}\left(1 - \frac{\alpha}{2} \right) \sqrt{s^2\left( 1 +  \matr{{X^*}^\top} (\matr{X}^\top \matr{X})^{-1}\matr{X^*}\right)} \right]
\end{align*}


\subsection*{Analyse de la variance}
\subsubsection*{Tableau ANOVA}
\begin{itemize}
\item On utilise le même tableau ANOVA qu'en régression linéaire simple.
\item $SSR_{\text{régression}} = \sum_{i=1}^{p} SSR_i $, où $SSR_i$ représente le SSR individuel de la variable explicative $i$ calculé par R. On peut ensuite trouver $MSR$ et la statistique $F_{obs}$.
\end{itemize}

\subsubsection*{Test F pour la validité globale de la régression}
Même test qu'en régression linéaire simple.


\subsubsection*{Test F partiel pour la réduction du modèle}
Avec $k < p$, on va rejeter
\begin{align*}
H_0 : Y_i = \beta_0 + \beta_1 x_{i1} + ... \beta_{ik} \quad \text{(modèle réduit)}
\end{align*}
Pour
\begin{align*}
H_1 : Y_i = \beta_0 + \beta_1 x_{i1} + ... \beta_{ip} \quad \text{(modèle complet)}
\end{align*}
Si
\begin{align*}
F_{obs} = \frac{(SSE^{(0)} - SSE^{(1)}) / \Delta dl}{SSE^{(1)} / (n-p')} \geq F_{p-k, n-p'}(1- \alpha)
\end{align*}
où $\Delta dl = p - k$, $SSE^{(0)}$ pour le modèle réduit ($H_0$) et $SSE^{(1)}$ pour le modèle complet ($H_1$).

\subsection*{Multicollinéarité}
\subsubsection*{Problèmes potentiels}
\begin{itemize}
\item Instabilité de $\mathrm{(X^\top X)^{-1}}$, i.e. une petite variation de $Y$ peut changer de grandes variations en $\hat{\beta}$ et $\hat{Y}$ ;
\item $\hat{\beta_i}$ de signes contre-intuitif ;
\item $Var(\hat{\beta_i})$ et $Var(\hat{Y})$ très grandes ;
\item Les méthodes de sélection de variable ne concordent pas ;
\item Conclusions erronées sur la significativité de certains paramètres, malgré une forte corrélation avec $Y$.
\end{itemize}

\subsubsection*{Détection}
\begin{itemize}
\item Si $r_{ij}$ dans la matrice de corrélation $\matr{{X^*}^\top X^*}$ est élevée, où $\matr{X^*} = \begin{bmatrix}
\frac{x_1 - \bar{x}_1}{s_1} & ... & \frac{x_p - \bar{x}_p}{s_p}
\end{bmatrix}_{1 \times p}$

\item Si le facteur d'influence de la variance ($VIF_j$) est élevé, où
\begin{align*}
VIF_j = \frac{1}{1 - R_j^2}
\end{align*}
avec $R_j^2$ le coefficient de détermination de la régression ayant comme variable réponse le $j$\up{e} variable et les $(j-1)$ autres variables exogènes en \textit{input}.
\item La variance de $\hat{\beta}_j$ s'exprime en fonction du VIF comme suit:
\begin{align*}
	\variance{\hat{\beta}_j} = \frac{\sigma^2}{(\matr{X^{*\top} \matr{X^*})_{jj}}} VIF_j
\end{align*}
\end{itemize}

\subsubsection*{Solution}
\begin{itemize}
	\item On retire les variables ayant un VIF élevé (une à la fois)
	\item On combine des variables exogènes redondantes	
\end{itemize}

\subsection*{Validation du modèle et des postulats}
\subsubsection*{Linéarité}
\begin{itemize}
\item On trace les graphiques à variable ajoutée ( $\hat{\varepsilon}_{\matr{Y} | \matr{X_{-j}}}$ en fonction de $\hat{\varepsilon}_{x_j | \matr{X_{-j}}}$).
\item Ces graphiques doivent normalement donner une droite de pente $\beta_j$.
\begin{itemize}
	\item Si le graphique ressemble à un graphique de résidus normaux, $x_j$ est inutile.
	\item Si il y a une courbe, $x_j$ est non-linéaire.
\end{itemize}
\end{itemize}

\subsubsection*{Homogénéité des variances}
\begin{itemize}
\item Graphique $r_i | \hat{Y}_i$
\end{itemize}

\subsubsection*{Indépendance entre les observations}
\begin{itemize}
\item Graphique $\hat{\varepsilon}_i | i$
\item Test de Durbin-Watson (pas à l'examen)
\end{itemize}


\section{Sélection de modèle et régression régularisée}
En présence de beaucoup de variable exogènes, on court le danger d'en garder trop ou pas assez
\begin{itemize}
	\item \textbf{Trop}: On augmente inutilement la variance des estimations$(\hat{\beta})$
	\item \textbf{Moins}: On augmente inutilement le biais des estimations$(\hat{\beta})$
\end{itemize}

\subsection*{Critères de comparaison classiques}
\begin{itemize}
	\item Coefficient de détermination (pour mesurer la qualité globale du modèle) :
	\[ R_2 = \frac{SSR}{SST} \]
	Si on ajoute une variable exogène, il est certain que $R^2$ augmentera, on utilise donc ce critère pour valider si la régression est utile pour prédire $Y$, mais pas pour critère de sélection des variables exogènes.
	\item Coeficient de détermination ajusté:
	\[ R_a^2 = \frac{SSE / p}{SST / (n-1)} = \frac{MSE}{MST} \]
	Ce critère permet de valider l'ajout de nouvelles variables exogènes.
\end{itemize}
Ces deux critères sont inutiles pour comparer des modèles avec des transformations différentes et pour des modèles avec/sans ordonnée à l'origine.

\subsection*{Méthode basées sur la puissance de prévision}
Ce critère maximise l'habileté du modèle a prédire de nouvelles données.
\subsubsection*{Principe de la validation croisée}
\begin{enumerate}
\item Pour $i = 1, ..., n$,
\begin{enumerate}[label=1.\arabic*]
	\item Enlever la $i$\up{e} observation du jeu de données.
	\item Estimer les paramètres du modèle à partir des $n - 1$ données restante.
	\item Prédire $Y_i$ à partir de $x_i$ et du modèle obtenu en 2, noté $\hat{Y}_{i,-i}$
\end{enumerate}
\item Calculer la somme des carrés des erreurs de prévision $PRESS = \sum_{i=1}^n (Y_i - \hat{Y}_{i,-i})^2$
\end{enumerate}
	On cherche a minimiser le PRESS ou à maximiser le coefficient de détermination de prévision:
	\[ R_p^2 = 1 - \frac{PRESS}{SST} \]
\subsubsection*{Les résidus PRESS}
	Il est possible de trouver la statistique PRESS sans devoir calculer $n$ régressions :
	\[ PRESS = \sum_{i=1}^n \left( \frac{\hat{\varepsilon_i}}{1 - h_{ii}} \right)^2 \]
\subsubsection*{Échantillion de test et validation croisée par $k$ ensemble}
\begin{enumerate}
\item Pour $k = 1, ..., K$,
\begin{enumerate}[label=1.\arabic*]
	\item Enlever le $k$\up{e} ensemble du jeu de donnée.
	\item  Estimer les paramètres du modèle à partir des données des $k - 1$ échantillons restants.
	\item Prédire les observations du $k$\up{e} ensemble $(\hat{Y}_{i,-k})$ et calculer
	\[ MSEP_k = \frac{1}{n_k} \sum_{i\in group\:k} (Y_i - \hat{Y}_{i,-k})^2 \]
\end{enumerate}
	\item Calculer la moyenne des sommes des carrés des erreurs de prévision $\frac{1}{k} \sum_{k=1}^k MSEP_k$
\end{enumerate}
	On choisit le modèle qui minimise $\frac{1}{k} \sum_{k=1}^k MSEP_k$


\subsection*{Le $C_p$ de Mallows}
\[ C_p = p' + \frac{(s_p^2 - \hat{\sigma^2})(n - p')}{\hat{\sigma^2}} = \frac{SSE}{\hat{\sigma^2}} + 2p' - n\]
On cherche le modèle pour lequel $C_p \approx p'$

\subsection*{Critère d'information d'akaike et critère bayésien de Schwarz}
\begin{itemize}
\item Ce critère est le plus utilisé dans la pratique et permet d'évaluer la qualité de l'ajustement d'un modèle. 
\[ AIC = n \cdot \ln \left(\frac{SSE}{n}\right) + 2p' \]
AIC prend en compte à la fois la qualité des prédictions du modèle et sa complexité.
\item BIC est similaire a AIC, mais la pénalité des paramètres dépend de la taille de l'échantillon.
On cherche à minimiser ces 2 critères.
\[ BIC = n \cdot \ln \left(\frac{SSE}{n}\right) + \ln(n)p' \]
\end{itemize}





\subsection*{Méthode algorithmiques}

\subsubsection*{Méthode d'inclusion (\textit{forward})}
\begin{enumerate}
\item On commence avec le modèle le plus simple (i.e. $\hat{Y}_i = \beta_0$)
\item On essaie d'ajouter la variable qui, en l'incluant dans le modèle, permet de réduire le plus le $SSE$ du modèle.
\item On valide si la variable diminue de façon significative les résidus avec un test $F$, où
\begin{align*}
F_{obs} = \frac{SSE_{\text{petit modèle}} - SSE_{\text{grand modèle}}}{SSE_{\text{grand modèle}} / (n-p')}
\end{align*}
On ajoute la variable au modèle si
\begin{align*}
F_{obs} > F_{1, n-p'}(1 - \alpha)
\end{align*}
\item On répète jusqu'à ce qu'aucune variable ne vaille la peine d'être ajoutée.
\end{enumerate}


\subsubsection*{Méthode d'exclusion (\textit{backward})}
\begin{enumerate}
\item On débute avec le modèle complet
\item On veut enlever la variable exogène qui, en l'excluant du modèle, permet de minimiser l'augmentation du $SSE$ de la régression.
\item Même test $F$ qu'à l'étape 3 de la méthode \textit{forward}, sauf qu'on enlève la variable seulement si
\begin{align*}
F_{obs} < F_{1, n-p'}(1-\alpha)
\end{align*}
\item On répète jusqu'à ce qu'aucune variable ne vaille la peine d'être enlevée.
\end{enumerate}


\subsubsection*{Méthode pas à pas (\textit{step-wise})}
\begin{enumerate}
\item On débute avec la méthode d'inclusion
\item Après l'ajout d'une variable au modèle, on effectue la méthode d'exclusion pour les variables qui sont actuellement dans le modèle (on remet constamment le modèle en question).
\end{enumerate}

\subsection*{Régression Ridge}
\begin{itemize}
\item Les coefficients de la régularisation sont réduits (\emph{shrinked}) car on applique une pénalité sur leur taille totale avec la norme $\ell_2 = \sqrt{\sum_{i=1}^{p} \beta_j^2}$
\item On veut minimiser l'équation suivante : 
\[R^{Ridge}(\beta) = \sum_{i=1}^{n} \left( Y_i - \beta_0 - \sum_{j=1}^{p} \beta_j x^{ij} \right)^2 + \lambda \sum_{j=1}^{p} \beta_j^2  \]
Et on trouve que
\[\hat{\bm{\beta}}^{Ridge} = \left( \matr{X}^\top Y + \lambda \matr{I}_{p \times p} \right)^{-1}  \matr{X}^\top \matr{Y}  \]
\item Cette méthode est très utile \textbf{Lorsqu'il y a beaucoup de variables explicatives}. On choisit la valeur optimale pour le coefficient de régularisation $\lambda$ avec une validation croisée.
\item Si la valeur de $\lambda$ augmente, le modèle perd en flexibilité et donc la variance des estimateurs diminue. Par contre, le biais augmente.
\item Le modèle de régression Ridge est plus difficile à interpréter, car plusieurs coefficients des paramètres peuvent être près de 0.
\end{itemize}

\subsection*{Régression Lasso (\emph{Least Absolute Shrinkage and Selection Operator})}
\begin{itemize}
\item Très similaire à la régression Ridge, sauf qu'on utilise la norme $\ell_1$ pour appliquer une contrainte à l'équation à minimiser : 
$\ell_1 = \sum_{j=1}^{p} | \beta_j | $

\item L'équation à minimiser est donc
\[S^{Lasso}(\bm{\beta})  = \sum_{i=1}^{n} \left( Y_i - \beta_0 - \sum_{j=1}^{p} \beta_j x_{ij} \right)^2 + \lambda \sum_{j=1}^{p} | \beta_j | \]

\item La différence avec Ridge est que les paramètres peuvent être égaux à zéro \textbf{(il y a donc une sélection des variables)}.
\end{itemize}



\section{Modèles linéaires généralisés (GLM)}
\subsection*{Famille exponentielle linéaire}
\subsubsection*{Définition}
Une loi de probabilité fait partie de la famille exponentielle linéaire si
\begin{enumerate}[label=\faAngleRight]
\item On peut exprimer la fonction de densité (ou masse) de probabilité comme
\[f(y ; \theta, \phi) =   \exp \left( \frac{y \theta - b(\theta)}{a(\phi)} + c(y ; \phi)    \right)  \] 
où $\theta$ est le paramètre canonique et $\phi$ est le paramètre de dispersion.

\item la fonction $c$ ne dépend pas du paramètre $\theta$.
\item Le support de $Y$ ne dépend pas des paramètres $\theta$ ou $\phi$.
\end{enumerate}

\subsubsection*{Propriétés}
Soit $\mu = \dot{b}(\theta) = \derivee{\theta} b(\theta)$ et $V(\mu) = \ddot{b}(\theta) = \frac{\partial^2}{\partial \theta^2} b(\theta)$. Alors, si $Y$ fait partie de la famille exponentielle linéaire, on peut exprimer l'espérance et la variance comme
\begin{align*}
\esp{Y}	& = \dot{b}(\theta) = \mu \\
\variance{Y}	& = a(\phi) \ddot{b}(\theta) = a(\phi) V(\mu) \\
\end{align*}

\subsubsection*{Lemme de la Log-vraisemblance}
Soit $\ell(\theta, \phi ; Y) = \mathrm{L}(\theta, \phi ; Y)$ la log-vraisemblance. Alors,
\[\esp{\derivee{\theta} \ell(\theta, \phi ; Y)} = 0\]
et
\[\esp{\left( \derivee{\theta}  \ell(\theta, \phi ; Y \right)^2} = - \esp{\frac{\partial^2}{\partial \theta^2} \ell(\theta, \phi ; Y)}\]

\subsection*{Fonction de lien}
Soit $\eta = \matr{X} \boldsymbol{\beta}$. La fonction de lien est la transformation qu'on applique à $\eta$ afin de limiter le support de $Y$.
\begin{description}
\item[Lien log] $\eta = \ln \mu \leftrightarrow \mu = e^{\eta}$
\item[Lien logistique] $\eta = \ln \left( \frac{\mu}{1 - \mu} \right) \leftrightarrow \mu = \frac{e^{\eta}}{1 + e^{\eta}}$

\item[Lien probit] $\eta = \Phi^{-1}(\mu) \leftrightarrow \mu = \Phi(\eta)$

\item[Lien log-log complémentaire] $\eta = \ln ( - \ln (1 - \mu)) \leftrightarrow \mu = 1 - e^{-e^{\eta}}$

\item[Lien canonique] $\eta = \theta$ 
\end{description}

\subsection*{Estimation des paramètres}
\begin{enumerate}[label=\faAngleRight]
\item On estime $\hat{\beta}$ avec la méthode du maximum de vraisemblance (EMV ou \emph{MLE} en anglais)

\item L'EMV est cohérent, i.e.
\[\hat{\bm{\beta}} \underset{n \to \infty}{\longrightarrow} \bm{\beta} \]

\item L'estimateur a une normalité asymptotique, i.e. lorsque $n \to \infty$,
\[  \hat{\bm{\beta}} \sim \mathcal{N} \left( \bm{\beta}, \frac{\mathcal{I}(\bm{\beta})^{-1}}{n} \right)    \]
où $\mathcal{I}(\bm{\beta})_{(p' \times p')}$ est la matrice d'information de Fisher : 
\begin{align*}
\mathcal{I}(\bm{\beta}) & = \esp{\dot{\ell}(\bm{\beta} ; Y_1, ..., Y_n) \dot{\ell}(\bm{\beta} ; Y_1, ..., Y_n)^{\top} } \\
	& = - \esp{\ddot{\ell}(\bm{\beta} ; Y_1, ..., Y_n)} \\
\end{align*}

\item On peut estimer la matrice d'information  de Fisher avec \textbf{l'information observée} :
\[\mathcal{I}(\hat{\bm{\beta}}) = - \sum_{i=1}^{n} \frac{\partial^2}{\partial \bm{\beta}^2} \ell(\bm{\beta} ; Y_i)  \eval_{\hat{\bm{\beta}}} \]
\end{enumerate}

\subsubsection*{Algorithme de Newton-Raphson}
L'objectif est de trouver $\hat{\bm{\beta}}$ qui maximise $\ell(\hat{\bm{\beta}})$, ce qui revient à trouver $\dot{\ell}(\hat{\bm{\beta}}) = 0$. On utilise l'approximation de Taylor de premier ordre dans l'algorithme : 
\begin{enumerate}[label = (\arabic*)]
\item Choisir des valeurs de départ pour le vecteur $\hat{\bm{\beta}}^{H_0}$
\item Pour $k = 1, 2, ...$
\begin{enumerate}[label = (2.\arabic*)]
	\item $\hat{\bm{\beta}}^{(k)} = \hat{\bm{\beta}}^{(k-1)} +  \left \{ - \ddot{\ell}(\hat{\bm{\beta}})^{(k-1)} \right \}^{-1} \dot{\ell}(\hat{\bm{\beta}})^{(k-1)}$
	\item Si $|\dot{\ell}(\hat{\bm{\beta}})^{(k)}| < \varepsilon $, on converge vers les paramètres optimaux pour le modèle et on arrète.
	\item Répéter les étapes (2.1) et (2.2) jusqu'à une convergence.
\end{enumerate}
\end{enumerate}

\subsubsection*{Méthode du score de Fisher}
Cette méthode est la même que l'algorithme de Newton-Raphson, à l'exception qu'on remplace $\mathcal{I} \left( \hat{\bm{\beta}} \right)$ par $- \esp{\ddot{\ell}(\hat{\bm{\beta}})}$ à l'étape (2.1)

\subsubsection*{Construction d'IC sur les paramètres}
\begin{itemize}
\item Lorsqu'on prédit des données, on peut aussi créer un I.C de confiance pour le prédicteur linéaire $\eta_i$. Par les propriétés du maximum de vraisemblance, quand $n \to \infty$, on a que $\hat{\bm{\beta}}$ est asymptotiquement normal. Alors, puisque $\eta$ est une combinaison linéaire de v.a. \emph{approximativement} normales, alors
\[\eta_i \approx \mathcal{N} \left( \eta_i,  \widehat{\mathrm{Var}}(\hat{\eta_i})   \right) \]

\item Et on a que (dans le cas simple où le modèle est $\beta_0 + \beta_1 x_{i1}$),
\begin{align*}
\variance{\hat{\eta}_i}	& = \variance{\hat{\beta}_0 + \hat{\beta}_1 x_{i1}} \\
& = \variance{\hat{\beta}_0} + x_{i1}^2 \variance{\hat{\beta}_1}  \\
& + 2 x _{i1}  \covar{\hat{\beta}_0, \hat{\beta}_1}
\end{align*}

\item Dans le cas multivarié, on a
\begin{align*}
\variance{\hat{\eta}_i}	 & = \matr{X}^{\top} \mathcal{I}(\hat{\bm{\beta}})^{-1} \matr{X}
\end{align*}

\item L'intervalle de confiance  pour $\eta_i$ est
\[\hat{\eta}_i  \pm z_{1- \alpha/2} \sqrt{\widehat{\mathrm{Var}}(\hat{\eta_i})} \]

\item Un intervalle de confiance (non-centré) pour $\mu_i$, en utilisant la fonction de lien inverse $g^{-1}(\eta)$ serait
\[ \mu_i \in \left [ g^{-1} \left( \hat{\eta}_i^{(L)}\right),  g^{-1} \left( \hat{\eta}_i^{(U)} \right)      \right]   \]

\item En utilisant la méthode Delta, on obtient un I.C qui est centré pour $\mu_i$, on a
\[\mu_i \in z_{1 - \frac{\alpha}{2}} \sqrt{\widehat{\mathrm{Var}}(\hat{\mu}_i)}  \]
où
\begin{align*}
\variance{\hat{\mu}_i}	& = \left( \derivee{\eta_i} g^{-1}(\eta_i) \eval_{\eta_i = \hat{\eta}_I} \right)^2 \variance{\hat{\eta}_i}
\end{align*}
\end{itemize}



\subsection*{Statistique de Wald}
Test d'hypothèse pour tester $H_0 : \beta_j = 0$, $H_1 : \beta_j \neq 0$. On a que
\[Z = \frac{\beta_j}{\sqrt{\widehat{\mathrm{Var}}(\hat{\beta}_j)}} \sim \mathcal{N}(0,1) \]
On rejète donc $H_0$ si $Z > z_{1 - \frac{\alpha}{2}}$.

\paragraph{Note} On obtient $\widehat{\mathrm{Var}}(\hat{\beta}_j)$ sur les éléments de la diagonale de $\{ \mathcal{I}(\hat{\beta}) \}^{-1} / n$.

\subsection*{Test du rapport de vraisemblance}
%% On pourrait peut-être synthétiser mieux cette section ...
On teste $H_0 : \beta \in \beta_0$ et $H_1 : \beta \in \beta_1$, où $\beta_1$ est le complément de l'espace $\beta_0$, qui est une sélection réduite des variables explicatives disponibles. On teste
\[\lambda(y) = \frac{\mathrm{L}\left(\hat{\beta}^{(H_0)} \right)}{\mathrm{L}(\hat{\beta})}   \]
$\lambda(y)$ sera assurément plus petit que 1 (il y a moins de variables explicatives). Mais on veut tester si $\lambda(y)$ est plus petit qu'une certaine valeur critique.

\begin{enumerate}[label=\faAngleRight]
\item Si \emph{$H_0$ spécifie tous les paramètres du modèle}, on a
\[-2 \ln \lambda(y) \sim \chi_{p'}^2 \quad , \text{Sous $H_0$}\]
\item Si \emph{$H_0$ spécifie partiellement les paramètres du modèle}, on a
\[-2 \ln \lambda(y) \sim \chi_{k_2 - k_1}^2 \quad , \text{Sous $H_0$}\]
où $k_1$ est le nombre de paramètres non-spécifiés dans $H_0$ et $k_2$ le nombre de paramètres non-spécifiés dans $H_1$.
\item Avec le TRV, on peut seulement comparer des modèles qui sont liés ($\hat{\beta}^{(H_0)}$ doit être un sous-ensemble de $\hat{\beta}$).
\end{enumerate}

\subsection*{Adéquation du modèle}

\subsubsection*{Statistiques $\chi^2$ de Pearson}
On peut valider l'adéquation du modèle avec la statistique $X^2$, où
\[  X^2 = \sum_{i=1}^{n} \left( \frac{y_i -  \hat{\mu}_i}{\sqrt{V(\hat{\mu}_i)}} \right)^2   \sim \chi_{n-p'}^2 \]
Avec $X^2 \leq \chi_{n-p', 1 - \frac{\alpha}{2}}^2$ si le modèle est adéquat. Si $\phi$ est inconnu, on peut l'estimer avec $\hat{\phi} = \frac{X^2}{n-p'}$

\subsubsection*{Déviance}
On a
\[2(\ell(\tilde{\theta}) - \ell(\hat{\theta}))  \sim \chi_{n-p'}^2 \]
avec $\bar{\theta}$ est le modèle nul, $\hat{\theta}$ le modèle à l'étude et $\tilde{\theta}$ le modèle complet, où $\hat{\mu}_i = y_i$. Cette expression représente la déviance $D(y ; \hat{\mu})$ : 
\begin{align*}
2(\ell(\tilde{\theta}) - \ell(\hat{\theta}))	& = 2 \sum_{i=1}^{n} \frac{w_i}{\phi} (y_i \tilde{\theta} - b(\tilde{\theta}) - y_i \hat{\theta} + b(\hat{\theta})) \\
&= 2 \sum_{i=1}^{n} \frac{w_i}{\phi} y_i (\tilde{\theta} - \hat{\theta}) - (b(\tilde{\theta} - b(\hat{\theta})) \\
& = \frac{D(y ; \hat{\mu})}{\phi}
\end{align*}
Si $\phi$ est inconnu, on peut l'estimer avec $\hat{\phi} = \frac{D(y ; \hat{\mu})}{n-p'}$

\subsection*{Comparaison de modèles}
Les critères classiques AIC et BIC peuvent être utilisés pour comparer des modèles. On peut aussi faire une analyse de la déviance

\subsubsection*{Analyse de la déviance}
\label{sssec:analyse_deviance}
On compare le modèle $A$ et le modèle $B$ (où $A$ est une simplification de $B$). Le modèle $A$ sera une bonne simplification de $B$ si
\[\frac{D(y ; \hat{\mu}_A) - D(y ; \hat{\mu}_B)}{\phi}  \sim \chi_{p_B - p_A}^2  \]
Il est certain que la déviance va augmenter en diminuant le nombre de paramètres. On veut valider si la déviance augmente \emph{significativement} au point de ne pas pouvoir simplifier $B$. On rejète $H_0$ que $A$ est une bonne simplification de $B$ si la différence est déviance réduite est supérieure à $\chi_{p_B - p_A, 1 - \frac{\alpha}{2}}^2$


\subsection*{Analyse des résidus}

\subsubsection*{Résidus de Pearson}
\begin{align*}
r_{P_i} = \frac{y_i - \hat{\mu}_i}{\sqrt{V(\hat{\mu})_i}}
\end{align*}
Aussi,  les résidus d'Anscombe et les résidus de la déviance.







% --- DÉBUT CHAPITRE 6 : MODÉLISATION DE DONNÉES DE COMPTAGE
\section{Modélisation de données de comptage}

\subsection*{Terme \emph{offset}}
On veut souvent modéliser le taux  de réclamation, cela se fait avec un terme \emph{offset} $t_i$ qui représente l'exposition au risque (i.e. le nombre d'années qu'on a assuré la personne) : 
\begin{align*}
\ln \left( \frac{\mu_i}{t_i} \right) & = \bm{x}_i \bm{\beta} \\
\ln (\mu_i) & = \bm{x}_i \bm{\beta} + \ln (t_i) \\
\mu_i & = t_i e^{\eta_i}
\end{align*}
le terme \emph{offset} peut être vu comme une variable explicative additionnelle (où le coefficient est toujours 1)

\subsection*{Notation pour les interactions}
Lorsqu'on utilise des variables catégoriques qui ont plusieurs niveaux, on peut utiliser une notation abbrégée. Prenons un modèle quelquonque \verb=A * B= avec la variable \verb=A= qui a I = 3 niveaux et \verb=B= qui a J = 2 niveaux. Alors, on aurait
\begin{align*}
\ln (\mu_{i,j}) = \alpha + \beta_i^A + \beta_j^B + \gamma_{i,j} \quad i = 1,2,3 \text{ et } j = 1,2
\end{align*}
0ù on impose les contraintes telles que $\beta_1^A = \beta_1^B = 0$ et $\gamma_{1,j} = \gamma_{j,1} = 0$. 

\subsection*{Approximation de la Binomiale par une Poisson}
Si la variable qu'on veut modéliser obéit à une $Bin(m, \pi)$ avec $m$ grand et $\pi$ petit, alors on peut l'approximer avec une loi de Poisson en prenant le modèle
\[\ln(\mu_i) = \ln(m_i) + \ln(\pi_i) \]
où $\ln(m_i)$ est un terme \emph{offset}

\subsection*{Tableau de contingence}
Lorsque toutes les variables sont des catégorielles, on peut créer un tableau de contingence, où on veut modéliser le nombre dans chaque case avec un GLM Poisson. \\

On a 3 modèles dans les tableaux de contingence (illustré avec des modèles simples qui ont les variables explicatives \verb=A=, \verb=B= et \verb=C= avec J,K et L niveaux  : 
\begin{enumerate}[label=\faAngleRight]
\item Modèle d'indépendance : $A + B + C$

\item Modèle d'indépendance partielle (celui qu'on veut tester) : 
\[A + B * C\]

\item Modèle d'indépendance conditionnelle (aussi appelé le \emph{modèle saturé}\footnote{Ce modèle est celui qui prédit le mieux, mais n'est d'aucune utilité car il a autant de paramètres qu'on a d'observations. On essaie donc de voir si le modèle d'indépendance partielle est une bonne simplification.} : 
\[A * B * C\]
\end{enumerate}
On peut alors tester l'indépendance de certaines variables en faisant une \textbf{Analyse de la déviance} (\autoref{sssec:analyse_deviance}).


\subsection*{Cote}
La cote de $A$ est définie par
\[\mathrm{Cote}(A) = \frac{\prob{A}}{\prob{\overline{A}}} = \frac{\prob{A}}{1 - \prob{A}} \]


\subsection*{Sousdispersion et susdispersion}
Avec le modèle Poisson, on suppose que $\esp{Y_i |x_i} = \variance{Y_i | x_i}$. Toutefois, les données peuvent être \textbf{sous-dispersées} si
\[  \esp{Y_i |x_i} > \variance{Y_i | x_i} \]
On détecte aussi la sous-dispersion si $D(y ; \hat{\mu}) / dl < 0.6$ ou $X^2 < 0.6$. On peut régler les problèmes de sous-dispersion en utilisant une distribution binomiale. Les données peuvent être \textbf{surdispersées} si
\[ \esp{Y_i |x_i} < \variance{Y_i | x_i}  \]
On le détecte lorsque $D(y ; \hat{\mu}) / dl > 1.7$ ou $X^2 > 1.7$

\subsection*{Binomiale négative}
Lorsque les données sont surdispersées, on peut utiliser la distribution binomiale négative dans notre modélisation. Soit $Y | Z = z \sim Pois(\mu z)$ et $Z \sim \Gamma(\theta_z, \theta_z)$, alors $\esp{Y} = \mu$ et $\variance{Y} = \mu + \frac{\mu^2}{\theta_z}$ et on a que $Y \sim BinNeg(\mu, \theta_z)$ telle que
\[f_Y(y) = \frac{\Gamma(\theta_z + y)}{\Gamma(\theta_z) y!} \left( \frac{\mu}{\mu + \theta_z} \right)^{y} \left( \frac{\theta_z}{\mu + \theta_z} \right)^{\theta_z}  \]
Lorsque $\theta_z \to \infty$, on retombe sur le modèle Poisson. On peut faire un TRV pour valider si le modèle Poisson est une bonne simplification du modèle binomiale négative : 
\[ \prob{ 2\Big(\ell^{Pois}(\bm{\hat{\beta}}) - \ell^{NB}(\bm{\hat{\beta}}) \Big) > x} = \frac{1}{2} \prob{\chi_{(1)}^2 > x} \] 

\subsection*{Modèle Poisson gonflée à zéro}
Lorsqu'on a une masse de probabilité à zéro plus importante à 0, on peut utiliser la loi de Poisson \emph{gonflée à zéro}, en modélisant à la fois la probabilité $\pi_i$ que la fréquence soit égale à zéro (avec un modèle binomial logistique) et $\lambda_i$ la fréquence avec un modèle Poisson avec fonction de lien log.


\section{Modélisation de données binomiales}





\subsection{Cas Bernouilli}

\subsubsection*{Tableau de mauvaise classification}
\begin{center}
\begin{tabular}{|c|c|c|}
\hline
  & \multicolumn{2}{c|}{Prédiction $\hat{Y}_i$} \\ \hline
Vrai $Y_i$ & 0 & 1 \\ 
  \hline
0 & $a$ & $b$ \\ 
  1 & $c$ & $d$ \\ 
   \hline
\end{tabular}
\end{center}
En forçant $\hat{Y}_i$ tel que
\begin{align*}
\hat{Y}_i = 
\begin{cases}
0	& , \hat{\pi}_i < \tau \\
1	& , \hat{\pi}_i \geq \tau \\
\end{cases}
\end{align*}
On peut calculer la statistique de \textbf{sensitivité} (i.e. le taux de bonne classificiation des vrais 1) et de \textbf{spécificité} (i.e. le taux de bonne classification des vrais 0) : 
\begin{align*}
\text{Sensitivité}	& = \alpha(\tau) = \frac{d}{c+d} \\
\text{Spécificité}	& = \beta(\tau) = \frac{a}{a+b} \\
\end{align*}






\end{multicols*}
%% -----------------------------
%% Fin du document
%% -----------------------------
\end{document}