%% Aide-mémoire
\documentclass[french, landscape]{article}
%% -----------------------------
%% Préambule
%% -----------------------------
\input{cheatsht-preamble.tex}

% Couleur de la page
%\pagecolor{gray!10!white}
\newcolumntype{a}{>{\columncolor{red!20!white}$}p{2cm}<{$}}

%% -----------------------------
%% Début du document
%% -----------------------------
\begin{document}

\small
\begin{multicols*}{3} % Nombre de colonnes (peut être changé plus tard.)
% \section{Rappel de probabilité}
% \subsection*{Certaines lois à savoir}
% \begin{tabular}{|a| * {4}{C|}}
% \hline
% \rowcolor{red!30!white}\text{Loi} & \prob{X = x} \text{ ou } f_X(x) & \esp{X} & Var(X) & M_X(t) \\\hline
% Bin(n,p)	& \binom{n}{x} p^x (1-p)^{n-x} & np & np(1-p) & \left( (1-p) + p^t \right)^n \\\hline
% Pois(\lambda) & \frac{e^{-\lambda} \lambda^x}{x!} & \lambda & \lambda & e^{\lambda(t-1)} \\\hline
% Gamma(\alpha, \lambda) & \frac{\lambda^{\alpha} x^{\alpha-1} e^{-\lambda x}}{\Gamma(\alpha)} & \frac{\alpha}{\lambda} & \frac{\alpha}{\lambda^2} & \left( \frac{\lambda}{\lambda - t} \right)^\alpha \\\hline
% Normale(\mu, \sigma^2) & \frac{1}{\sqrt{2 \pi} \sigma} e^{- \frac{1}{2} \left( \frac{x-\mu}{\sigma} \right)^2} & \mu & \sigma^2 & e^{\mu t + \frac{\sigma^2 t^2}{2}} \\\hline
% \end{tabular}

\section{Estimation non-paramétrique}
\subsection*{Moments à savoir}
\begin{align*}
\mu_k^{\prime} 	& = \esp{X^k} \\
\mu_k			& = \esp{(X-\mu)^k} \\
CV				& = \frac{\sigma}{\esp{X}} \\
\gamma_1			& = \frac{\mu_3}{\sigma^3}  \\
\gamma_2			& = \frac{\mu_4}{\sigma^4} \\
\end{align*}

\subsection*{Fonction empirique}
\begin{align*}
F_n(x) &= \frac{1}{n} \sum\limits_{j=1}^n I_{\{ x_j \leq x\}} \\
f_n(x) &= \frac{1}{n} \sum\limits_{j=1}^n I_{\{ x_j = x\}}  \\
 n F_n&(x)  \sim \text{bin}(n, F(x))\\
E[F_n(x)] &= \frac{n F_n(x)}{n} = F_n(x)\\
\widehat{Var}[F_n(x)] &= \frac{n F_n(x)(1 - F)n(x)}{n^2} \\
       &= \frac{F_n(x)(1 - F_n(x))}{n}  \\
\widehat{Var}[S_n(x)] &= \frac{S_n(x)(1 - S_n(x))}{n} \\
F_n(x) &= 
\left\{
	\begin{array}{ll}
		0,  &  x < y_1 \\
        1 - \frac{r_j}{n}, &  y_{j-1} \leq x < y_j, j=2,...,k \\
        1, & x > y_k 
	\end{array}
\right.
\end{align*}

\subsection*{Estimateur de Nelson-Aalen}
\begin{itemize}
    \item L'estimateur de Nelson–Aalen est une alternative à la fonction de répartition empirique comme estimateur de la fonction de survie dans le cas de données complètes.
\end{itemize}
\begin{align*}
    h(x) &= -ln S(x) \\
    \hat{H}(x) &= 
    \left\{
	\begin{array}{ll}
		0,  &  x < y_1 \\
        \sum\limits_{i=1}^{j-1} \frac{s_i}{r_i}, &  y_{j-1} \leq x < y_j, j=2,...,k \\
        \sum\limits_{i=1}^{k} \frac{s_i}{r_i}, & x > y_k 
	\end{array}
\right. \\
\hat{S}(x) &= e^{-\hat{H}(x)} \\
\frac{s_i}{r_i} &= \frac{s_i/n}{r_i/n} \\
                &= \frac{s_i/n}{1 - (1 - F_n(y_{i-1}))} \\
                &= \frac{f_n(y_i)}{1 - (1 - F_n(y_{i-1}))} \\
E[\hat{H}(y_j)] &= H(y_j) \\
\widehat{Var}[\hat{H}(y_j)] &\approx \sum_{i=1}^j \frac{s_i}{r_i^2} 
\end{align*}

\subsection*{Estimateur de Kaplan-Meier}
\begin{align*}
    r_i &= \text{\# sujets sortis du groupe à ou après $y_i$} \\
        &= \text{\# sujets non encore entrés dans le groupe à $y_i$} \\
    \hat{S}(x) &=
    \left\{
	\begin{array}{ll}
		1,  &  0 \leq x < y_1 \\
        \prod\limits_{i=1}^{j-1} \frac{r_i - s_i}{r_i}, &  y_{j-1} \leq x < y_j, j=2,...,k \\
        \prod\limits_{i=1}^{k} \frac{r_i - s_i}{r_i}, & x > y_k \: \text{(0 si données complètes)}
	\end{array}
    \right. \\
    E[\hat{S}(x)] &= \frac{S(y_i)}{S(y_1)}\: \text{i.e. sans biais à $y_i$} \\
    \widehat{Var}[\hat{S}(y_j)] &= [\hat{S}(y_j)]^2 \sum_{i=1}^j \frac{s_i}{r_i(r_i - s_i)}\: \textbf{(Aprox. GreenWood)}
\end{align*}
\end{multicols*}

%% -----------------------------
%% Fin du document
%% -----------------------------
\end{document}