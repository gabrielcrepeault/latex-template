% Template pour faire aide-mémoire
\documentclass[10pt, french]{article}

%% -----------------------------
%% Préambule
%% -----------------------------
\input{cheatsht-preamble.tex}

%% -----------------------------
%% Début du document
%% -----------------------------
\begin{document}



\footnotesize % Écrire petit (peut être modifié)
\begin{multicols*}{3} % Nombre de colonnes (peut être changé plus tard.)
\section{Avant l'Examen 1}
À ajouter plus tard (prendre ce que Nich avait déjà fait avant l'intra)


\section{Contrats d'assurance-vie}
Le paiement est soit en continu, soit à la fin de l'année ou à la fin de la $\frac{1}{m}$ d'année.

\paragraph{Assurance-vie entière} On verse le capital au décès de l'assuré

\begin{flalign*}
\Ax*{x} & = \int_{0}^{\omega - x} v^t \px[t]{x} \mu_{x+t} dt \\
\Ax{x}	& = \sum_{k=0}^{\omega - x - 1} v^{k+1} \qx[k|]{x} \\
	& = \sum_{k=0}^{\omega - x - 1} v^{k+1} \px[k]{x} \qx[]{x+k} \\
\end{flalign*}

\paragraph{Assurance-vie temporaire} On verse le capital au décès de l'assuré, s'il survient dans les $n$ prochaines années.
\begin{flalign*}
\Ax*{\termxn}	& = \int_{0}^{n} v^t \px[t]{x} \mu_{x+t} dt \\
\Ax{\termxn}		& = \sum_{k=0}^{n-1} v^{k+1} \qx[k|]{x} \\
	& = \sum_{k=0}^{n-1} v^{k+1} \px[k]{x} \qx[]{x+k} \\
\end{flalign*}

\paragraph{Assurance-vie dotation pure} On verse le capital à l'assuré si celui-ci est toujours en vie après $n$ années.
\begin{align*}
\Ax{\pureendowxn}	& = \px[n]{x} v^n
\end{align*}

\paragraph{Assurance mixte} On verse le capital à l'assuré si il décède dans les $n$ prochaines années, ou si il est toujours en vie après cette période.
\begin{align*}
\Ax*{x:\angln}	& = \int_{0}^{n} v^t \px[t]{x} \mu_{x+t} dt + v^n \px[n]{x} \\
	& = \Ax*{\termxn} + \Ax{\pureendowxn} \\
\Ax{x:\angln}		& = \sum_{k=0}^{n-1} v^{k+1} \qx[k|]{x} + v^n \px[n]{x} \\
\end{align*}

\paragraph{Assurance différée} On verse le capital à l'assuré lors de son décès seulement si le décès survient dans plus de $m$ années \footnote{Interprétation : Une assurance-vie entière qui débute dans $m$ années.}

\begin{align*}
\Ax*[m|]{x}	& = \int_{m}^{\omega -x} v^t \px[t]{x} \mu_{x+t} dt \\
	& = v^m \px[m]{x} \int_{0}^{\omega - x - m} v^t \px[t]{x+m} \mu_{(x+m)+t} dt \\
	& = \Ex[m]{x} \Ax*{x+m} \\
\Ax[m|]{x}	& = \sum_{k=m}^{\omega - x -1} v^{k+1} \qx[k|]{x} \\
	& = \sum_{k=0}^{\omega - x - m - 1} v^{k+1+m} \qx[(k+m)|]{x} \\
	& = v^m \px[m]{x} \sum_{k=0}^{\omega - (x+m) - 1} v^{k+1} \px[k]{x+m} \qx[]{x+m+k} \\
	& = \Ex[m]{x} \Ax{x+m} \\
\end{align*}
où $\Ex[m]{x}$ est un facteur d'actualisation actuarielle.






\end{multicols*}
%% -----------------------------
%% Fin du document
%% -----------------------------
\end{document}